\documentclass{article}

\usepackage{fullpage,url}

\newcommand{\CC}{C\nolinebreak\hspace{-.05em}\raisebox{.4ex}{\tiny\bf +}\nolinebreak\hspace{-.10em}\raisebox{.4ex}{\tiny\bf +}}

\title{Minimum Degree and Fill-in Heuristics}
\author{Eli Fox-Epstein}

\begin{document}
\maketitle
\thispagestyle{empty}

This abstract corresponds to the submission for the 2016 PACE Challenge for
Track A. This is a serial, heuristic implementation.  Source code is available
at: \url{https://github.com/elitheeli/2016-pace-challenge}.

This entry is provided as a baseline, implementing a simple strategy: the
Minimum Degree Heuristic~\cite{markowitz1957elimination} and Minimum Fill-In
Heuristic. In the time given, this implementation first quickly finds a tree
decomposition with the Minimum Degree Heuristic and then runs the Minimum
Fill-In Heuristic repeatedly, maintaining the best decomposition.

This implementation is written in \CC{}11. Tree decompositions are represented
by two vectors: one is a vector of bags (each of which is a vector of vertex
IDs) and one stores parent-child relationships.

The graph is represented using an adjacency matrix (compactly represented as a
{\tt std::vector<bool>}), as well as a adjacency list for each vertex (using a
{\tt std::vector}) to support efficient neighborhood iteration and a vector
tracking vertex degrees.  At any given time, the adjacency lists contain a
\emph{superset} of the neighbors of a vertex: on iteration through a
neighborhood, each potential neighbor is verified with the matrix and discarded
if it is not a true neighbor.

The implementation is templated by a function of the graph and a vertex 
that returns an floating point number.  The algorithm consists of a number of
iterations, each of which selects the vertex minimizing the function, adds
edges to turn its neighborhood into a clique, and deletes the vertex from the
graph.  The order of vertex selection is a perfect elimination ordering in the
graph consisting of all edges that were ever in the graph.
  
This template is instantiated for functions that return the degree of the
vertex (for the Minimum Degree Heuristic) and the number of edges missing in
its neighborhood (for the Minimum Fill-In Heuristic).  Internally, the
algorithm maintains a priority queue on the vertices, keyed by the templated
function plus a small amount of random noise. The noise ensures a random choice
among the vertices minimizing the function. Three additional vectors are
maintained: one to store the order in which vertices are selected, one mapping
a vertex to whether or not it has been selected, and one tracking the function
value of a vertex when last enqueued.  Initially, all vertices are enqueued.
When processing a vertex~$v$, if~$v$ has already been seen, it is simply
discarded.  If its key is less than its current function value, it is
re-enqueued with the correct key.  Finally, if its key is correct, it is added
to the ordering, marked as visited, and its neighbors are connected into a
clique and then disconnected from~$v$. Some neighbors may be re-enqueued if
their function values decreased.

\bibliographystyle{plain}
\bibliography{bibliography}

\end{document}
